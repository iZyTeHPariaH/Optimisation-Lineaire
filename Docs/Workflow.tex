\documentclass{article}
\usepackage[utf8]{inputenc}
\usepackage[T1]{fontenc}
\usepackage[cyr]{aeguill}



\title{Modélisation d'un Workflow par un programme mathématique}

\begin{document}
 \maketitle
 \section{Introduction}
 Avantages : 
\begin{itemize}
  \item Solution optimale
  \item Algorithme de résolution courant
  \item Réoptimisation et modification du modèle ``simples''
\end{itemize}

 \section{Modélisation}
 \subsection{Données}
On dispose des données suivantes :
\begin{itemize}
  \item $d_{i,j}$ : durée de la tâche $i$ si nous l'executons sur la machine $j$
  \item $D_{i,j}$ : durée du transfert de l'output de la tâche $i$ vers la machine $j$
\end{itemize}
 \subsection{Modèle}
Le coeur de notre modélisation se situe dans les variables $r_{i,j,k}$ définies
$$
r_{i,j,k} = \left\{\begin{array}{ll}
                   1 & \textrm{si la tâche i est executée par la machine j à l'instant k} \\
                   0 & sinon
                  \end{array}
            \right.
$$

On introduit également des variables auxiliaires : 
\begin{itemize}
  \item $x_i$ : date de début de la tâche $i$
  \item $y_{i,j} = \left\{\begin{array}{ll}
                            1 & \textrm{si la tâche i est executée sur la machine j}\\
                            0 & sinon
                          \end{array}
                   \right.$
  \item $d_i$ : durée de la tâche $i$ fixée
\end{itemize}
\end{document}
